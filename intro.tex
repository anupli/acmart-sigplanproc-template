\section{Introduction}

Structure the introduction along the same lines as the abstract.
Expand on the problem, contribution, result, and meaning in one or two
paragraphs on each.

% problem

% contribution

% result

% meaning

A few grammatical notes.  Elminate passive voice.  Passivity
introduces ambiguity (Who is the actor?).  Ambiguity is occasionally a
useful device, but as a rule it is the antithesis of effective
scientific communication, so should be avoided.  Do not use a citation
as a noun, always use it as a footnote.  For example, instead of
``\cite{BH:04} measure barrier overheads.''  say ``\citeauthor{BH:04}
measure barrier overheads~\cite{BH:04}.''  Do not use an unqualified
``this'' as the noun of your sentence. For example, instead of ``This
makes caches rock.''  say ``This locality of reference makes caches
rock.''

Some other points of style.  When writing numbers with units, you
should insert a small space (\textjava{\\,}) between the number and
the units.  For example, use `\textjava{2.8\\,GHz}', which renders as
2.8\,GHz.  The space is there for correctness (see NIST standards on
units), and the small space both typesets it nicely and ensures the
unit does not wrap onto a different line from the number.

The \textjava{todonotes} package provides nice
\todo[fancyline,color=green,size=\footnotesize]{This is a test of
  todonotes} tricks for adding todo items (\textjava{\\todo}) and
notes to your draft document.  You can make macros with \kathryn{This
  is a note from Kathryn.} color-coded notes from each of the authors.
When you are ready to submit, you should silence them all with the
\textjava{disable} package option.  Figure~\ref{fig:todo}
shows\steve{Here's one from Steve.}  the package's
\textjava{\\missingfigure} macro, which is a nice way to make
place-holders.

When there are numbers for checking before submit, you can use \pending{12456}
to highlight the number. After you check the number, switch it to
\checked{12456}. When you submit the paper, change the these two macros to submit
version, then the color will disappear.

\begin{figure}
  \centering
  \missingfigure{There's a figure missing here}
  \caption{An example of a missing figure, using \textjava{todonotes}'
    \textjava{\\missingfigure} macro.}
  \label{fig:todo}
\end{figure}

There are good tools for automating the process of running latex.  I
discourage you from using a Makefile.  If your editor does not do the
job for you, you should probably use \textjava{latexmk}, which is
bundled as part of the standard texlive distribution.  There is an
example configuration file provided in this directory.  Copy it into
your home director with the name \textjava{.latexmkrc} (notice the
dot).

Gratuitous citation to generate a few bib entries \cite{BH:04,SBF:12,YBFH:12}.

%%% Local Variables:
%%% mode: latex
%%% TeX-master: "paper"
%%% End:
